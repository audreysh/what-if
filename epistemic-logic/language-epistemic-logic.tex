% Part: applied-modal-logics
% Chapter: epistemic-logic
% Section: language-epistemic-logic

\documentclass[../../../include/open-logic-section]{subfiles}

\begin{document}

\olfileid{aml}{el}{lan}

\olsection{The Language of Epistemic Logic}

\begin{defn}
Let $A$ be a set of agent-symbols. The basic language of multi-agent epistemic logic contains
\begin{enumerate}
  \tagitem{prvFalse}{The propositional constant for !!{falsity}~$\lfalse$.}{}
  \tagitem{prvTrue}{The propositional constant for !!{truth}~$\ltrue$.}{}
  \item A !!{denumerable}s set of !!{propositional variable}s: $\Obj
    p_0$, $\Obj p_1$, $\Obj p_2$, \dots
  \item The propositional connectives: \startycommalist
  \iftag{prvNot}{\ycomma $\lnot$ (negation)}{}%
  \iftag{prvAnd}{\ycomma $\land$ (conjunction)}{}%
  \iftag{prvOr}{\ycomma $\lor$ (disjunction)}{}%
  \iftag{prvIf}{\ycomma $\lif$ (!!{conditional})}{}%
  \iftag{prvIff}{\ycomma $\liff$ (!!{biconditional})}{}.
  \item The knowledge operator $\Knows_a$ where $a \in A$.
\end{enumerate}
\end{defn}

If we are only concerned with the knowledge of a single agent in our
system, we can drop the reference to the set~$A$, and individual
agents. In that case, we only have the basic operator~$\Knows$.

\begin{defn}
\emph{!!^{formula}s} of the epistemic language are inductively
  defined as follows:
\begin{enumerate}
\tagitem{prvFalse}{$\lfalse$ is an atomic !!{formula}.}{}

\tagitem{prvTrue}{$\ltrue$ is an atomic !!{formula}.}{}

\item Every propositional variable $\Obj p_i$ is an (atomic) !!{formula}.

\tagitem{prvNot}{If $!A$ is !!a{formula}, then $\lnot !A$ is
  !!a{formula}.}{}

\tagitem{prvAnd}{If $!A$ and $!B$ are !!{formula}s, then $(!A \land
  !B)$ is !!a{formula}.}{}

\tagitem{prvOr}{If $!A$ and $!B$ are !!{formula}s, then $(!A \lor !B)$
  is !!a{formula}.}{}

\tagitem{prvIf}{If $!A$ and $!B$ are !!{formula}s, then $(!A \lif !B)$
  is !!a{formula}.}{}

\tagitem{prvIff}{If $!A$ and $!B$ are !!{formula}s, then $(!A \liff !B)$
  is !!a{formula}.}{}

\item If $!A$ is !!a{formula} and $a \in A$, then $\Knows_a !A$ is
  !!a{formula}.

\tagitem{limitClause}{Nothing else is a !!{formula}.}{}
\end{enumerate}
\end{defn}

If !!a{formula}~$!A$ does not contain $\Knows_a$, we say it
is \emph{modal-free}.

\begin{defn}
  While the $\Knows$ operator is intended to symbolize individual
  knowledge, $\EKnows$, often read as ``everybody knows,'' symbolizes
  group knowledge. Where $B \subseteq A$, we define $\EKnows_B !A$
 as an abbreviation for \[\bigwedge_{b \in B} \Knows_b !A.\]
\end{defn}

We can also define an even stronger sense of knowledge, namely
\emph{common knowledge} among a group of agents~$B$. When a piece of
information is common knowledge among a group of agents, it means that
for every combination of agents in that group, they all know that each
other knows that each other knows \dots ad infinitum. This is
significantly stronger than group knowledge, and it is easy to come up
with relational models in which !!a{formula} is group knowledge, but
not common knowledge. We will use $\CKnows_B !A$ to symbolize ``it is
common knowledge among~$B$ that~$!A$.''

\end{document}
