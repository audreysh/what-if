% Part: applied-modal-logics
% Chapter: epistemic-logic
% Section: relational-models

\documentclass[../../../include/open-logic-section]{subfiles}

\begin{document}

\olfileid{aml}{el}{rel}

\olsection{Relational Models}

The basic semantic concept for epistemic logics is the same as that of
ordinary modal logics. Relational models still consist of a set of
worlds, and an assignment that determines which !!{propositional
variable}s count as ``true'' at which worlds. And if we are only
dealing with a single agent, we have a single accessibility relation
as usual. However, if we have a multi-agent epistemic logic, then our
single accessibility relation becomes a set of accessibility
relations, one for each~$a$ in our set of agent symbols~$G$.

A \emph{relational model} consists of a set of worlds, which are
related by binary accessibility relations---one for each
agent---together with an assignment which determines which
!!{propositional variable}s are true at which worlds.

\begin{defn}
  A \emph{model} for the multi-agent epistemic language is a triple
  $\mModel{M} = \tuple{W, R, V}$, where
  \begin{enumerate}
  \item $W$ is a nonempty set of ``worlds,''
  \item For each $a \in G$, ${R}_a$ is a binary accessibility relation
  on~$W$, and
  \item $V$ is a function assigning to each !!{propositional
    variable}~$p$ a set $V(p)$ of possible worlds.
  \end{enumerate}
  When $R_a ww'$ holds, we say that $w'$ is \emph{accessible by a
    from}~$w$. When $w \in V(p)$ we say $p$ is \emph{true at}~$w$.
\end{defn}

The mechanics are just like the mechanics for normal modal logic, just
with more accessibility relations added in. For a given agent, we will
generally interpret their accessibility relation as representing
something about their informational states. For example, we often
treat $R_a ww'$, as expressing that $w'$ is consistent with~$a$'s
information at~$w$. Or to put it another way, at~$w$, they cannot tell
the difference between world $w$ and world~$w'$.  

\end{document}
