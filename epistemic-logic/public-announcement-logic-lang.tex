% Part: applied-modal-logics
% Chapter: epistemic-logic
% Section: language-epistemic-logic

\documentclass[../../../include/open-logic-section]{subfiles}

\begin{document}

\olfileid{aml}{el}{pal}

\olsection{Public Announcement Logic}

Dynamic epistemic logics allow us to represent the ways in which agents' knowledge changes over time,
or as they gain new information. Many of these represent changes in knowlege using informational
\emph{events} or \emph{updates}. The most basic kind of update is a public announcement in which some
formula is truthfully announced and all of the agents witness this taking place together. To do this,
we expand the language as follows

\begin{defn}
Let $G$ be a set of agent-symbols. The basic language of multi-agent epistemic logic with 
public announcements contains
\begin{enumerate}
  \tagitem{prvFalse}{The propositional constant for !!{falsity}~$\lfalse$.}{}
  \tagitem{prvTrue}{The propositional constant for !!{truth}~$\ltrue$.}{}
  \item A !!{denumerable}s set of !!{propositional variable}s: $\Obj
    p_0$, $\Obj p_1$, $\Obj p_2$, \dots
  \item The propositional connectives: \startycommalist
  \iftag{prvNot}{\ycomma $\lnot$ (negation)}{}%
  \iftag{prvAnd}{\ycomma $\land$ (conjunction)}{}%
  \iftag{prvOr}{\ycomma $\lor$ (disjunction)}{}%
  \iftag{prvIf}{\ycomma $\lif$ (!!{conditional})}{}%
  \item The knowledge operator $\Knows_a$ where $a \in G$.
  \item The public announcement operator $[!B]$ where $!B$ is a !!{formula}.
\end{enumerate}
\end{defn}

The public announcement operator functions as a box operator, and our inductive definition of the language is given accordingly:

\begin{defn}
\emph{!!^{formula}s} of the epistemic language are inductively
  defined as follows:
\begin{enumerate}
\tagitem{prvFalse}{$\lfalse$ is an atomic !!{formula}.}{}

\tagitem{prvTrue}{$\ltrue$ is an atomic !!{formula}.}{}

\item Every propositional variable $\Obj p_i$ is an (atomic) !!{formula}.

\tagitem{prvNot}{If $!A$ is !!a{formula}, then $\lnot !A$ is
  !!a{formula}.}{}

\tagitem{prvAnd}{If $!A$ and $!B$ are !!{formula}s, then $(!A \land
  !B)$ is !!a{formula}.}{}

\tagitem{prvOr}{If $!A$ and $!B$ are !!{formula}s, then $(!A \lor !B)$
  is !!a{formula}.}{}

\tagitem{prvIf}{If $!A$ and $!B$ are !!{formula}s, then $(!A \lif !B)$
  is !!a{formula}.}{}

\tagitem{prvIff}{If $!A$ and $!B$ are !!{formula}s, then $(!A \liff !B)$
  is !!a{formula}.}{}

\item If $!A$ is !!a{formula} and $a \in G$, then $\Knows_a !A$ is
  !!a{formula}.
  
\item If $!A$ and $!B$ are !!{formula}s, then $[!A] !B$ is !!a{formula}.

\tagitem{limitClause}{Nothing else is a !!{formula}.}{}
\end{enumerate}
\end{defn}

The intended reading of the !!{formula} $[!A] !B$ is ``After $!A$ is truthfully announced,
$B$ holds. It will sometimes also be useful to talk about common knowledge in the context
of public announcements, so the language may also include the common knowledge
operator $\CKnows_G !A$.

\end{document}
