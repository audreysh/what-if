% Part: applied-modal-logics
% Chapter: epistemic-logic
% Section: public-announcement-logic-semantics

\documentclass[../../../include/open-logic-section]{subfiles}

\begin{document}

\olfileid{aml}{el}{pal}

\olsection{Semantics of Public Announcement Logic}

Relational models for public announcement logics are the same as they were in epistemic logics. However,
the semantics for the public announcement operator are something new.

\begin{defn}\ollabel{defn:mmodels}
  \emph{Truth of !!a{formula}~$!A$ at~$w$} in a~$\mModel M = \langle W, R, V \rangle$, in symbols:
  $\mSat{M}{!A}[w]$, is defined inductively as follows:
  \begin{enumerate}
  \tagitem{prvFalse}{\indcase{!A}{\lfalse}{Never $\mSat{M}{\lfalse}[w]$}.}{}
  \tagitem{prvTrue}{\indcase{!A}{\ltrue}{Always $\mSat{M}{\ltrue}[w]$}.}{}
  \item $\mSat{M}{p}[w]$ iff $w \in V(p)$
  \tagitem{prvNot}{\indcase{!A}{\lnot !B}{$\mSat{M}{\indfrm}[w]$ iff
    $\mSat/{M}{!B}[w]$}.}{}
  \tagitem{prvAnd}{\indcase{!A}{(!B \land !C)}{$\mSat{M}{\indfrm}[w]$ iff
    $\mSat{M}{!B}[w]$ and $\mSat{M}{!C}[w]$}.}{}
  \tagitem{prvOr}{\indcase{!A}{(!B \lor !C)}{$\mSat{M}{\indfrm}[w]$ iff
    $\mSat{M}{!B}[w]$ or $\mSat{M}{!C}[w]$} (or both).}{}
  \tagitem{prvIf}{\indcase{!A}{(!B \lif !C)}{$\mSat{M}{\indfrm}[w]$ iff
    $\mSat/{M}{!B}[w]$ or $\mSat{M}{!C}[w]$}.}{}
  \tagitem{prvIff}{\indcase{!A}{(!B \liff !C)}{$\mSat{M}{\indfrm}[w]$ iff
    either both $\mSat{M}{!B}[w]$ and $\mSat{M}{!C}[w]$ or
    neither $\mSat{M}{!B}[w]$ nor $\mSat{M}{!C}[w]$}.}{}
  \item\ollabel{defn:sub:mmodels-box}
    \indcase{!A}{\Knows_a !B}{$\mSat{M}{\indfrm}[w]$ iff
    $\mSat{M}{!B}[w']$ for all $w' \in W$ with $R_a ww'$}
  \item\ollabel{defn:sub:mmodels-pal}
    \indcase{!A}{[!B] !C}{$\mSat{M}{\indfrm}[w]$ iff
    $\mSat{M}{!B}[w]$ implies $\mSat{M \mid !B}{!C}[w]$}

  
  Where $M \mid !B = \langle W', R', V' \rangle$ is defined as follows:
  
  \begin{enumerate}
  \item $W' = \{ u \in W \mid \mSat{M}{!B}[u] \}$. 
  So the worlds of $M \mid !B$ are the worlds in $M$ at which $!B$ holds.
  \item $R'_a = R_a \cap (W' \times W')$.
  Each agent's accessibility relation is simply restricted to the worlds that remain in $W'$.
  \item $V'(p) = \{ u \in W' \mid u \in V(p) \}$.
  Similarly, the propositional valuations at worlds remain the same, representing the idea that informational events
  will not change the truth value of propositional variables.
  \end{enumerate}
  
  \end{enumerate} 
\end{defn}

What is distinctive, then, about public announcement logics, is that the truth of !!a{formula} at $M$
can sometimes only be decided by referring to a model other than $M$ itself.

Notice also that our semantics treats the announcement operator as a $\Box$ operator, and so
if !!a{formula} $!A$ cannot be truthfully announced at a world, then $[!A]B$ will hold there trivially,
just as all $\Box$ !!{formula}s hold at endpoints. 

\begin{figure}
  \begin{center}
    \begin{tikzpicture}[modal]
      \node[world] (w1) [label={below left:$p, \lnot q$}]{$w_1$}; 
      \node[world] (w2) [left=of w1, label={left:$\lnot p, \lnot q$}]{$w_2$}; 
      \node[world] (w3) [above=of w1, label={right:$p, q$}] {$w_3$};
      \draw[<->] (w1) to [swap] node {$b$} (w2);
      \draw[->,loop below] (w1) to node {$a, b$} (w1);
      \draw[<->] (w1) to [swap] node {$a$} (w3);
      \draw[->,loop above] (w2) to node {$a, b$} (w2) ;
      \draw[->,loop above] (w3) to node {$a, b$} (w3) ;
      
      \node[world](v1)[right of=w1, xshift=1.25in, label={right:$p, \lnot q$}]{$w'_1$};
      \node[world](v3)[above of=v1,label={right:$p,q$}]{$w'_3$};
      \draw[<->] (v1) to node {$a$} (v3);
      \draw[->,loop below] (v1) to node {$a,b$} (v1);
      \draw[->,loop above] (v3) to node {$a,b$} (v3) ;
      
      \node(m)[below=of w1]{$M$};
      \node(m')[below=of v1]{$M \mid p$}; 

      \draw[-,dotted] (w1) to node {\footnotesize announcement of $p$}(v1) ;
     
    \end{tikzpicture}
  \end{center}
  \caption{Before and after the public announcement of $p$.}
  \ollabel{fig:announcement-example}
\end{figure}

We can see the public announcement of a !!{formula} as shrinking a model, or restricting
it to the worlds at which the !!{formula} was true. \olref{fig:announcement-example} gives an
example of the effects of publicly announcing $p$. One notable thing about that model is that
agent $b$ learns that $p$ as a result of the announcement, while agent $a$ does not (since $a$
already knew that $p$ was true).

More formally, we have $\mSat{M}{\lnot \Knows_b p}[w_1]$ but $\mSat{M \mid p}{\Knows_b p}[w'_1]$. 
This implies that $\mSat{M}{[p] \Knows_b p}[w_1]$. But we have some even stronger claims that we 
can make about the result of the announcement. In fact, it is the case that $\mSat{M}{[p]\CKnows_{\{a,b\}} p}[w_1]$.
In other words, after $p$ is announced, it becomes \emph{common knowledge}.

We might wonder, though, whether this holds in the general case, and whether a truthful
announcement of $!A$ will \emph{always} result in $!A$ becoming common knowledge. It may
be surprising that the answer is in fact no. And in fact, it is possible to truthfully announce formulas that will
no longer be true once they are announced. For example, consider the effects of announcing 
$p \land \lnot \Knows_b p$ at $w_1$ in \olref{fig:announcement-example}. In fact, $M \mid p$ and
$M \mid (p \land \lnot \Knows_b p)$ are the same model. However, as we have already noted, 
$\mSat{M \mid p}{\Knows_b p}[w'_1]$. Therefore,
 $\mSat{M \mid (p \land \lnot \Knows_b p)}{\lnot (p \land \lnot \Knows_b p)}[w'_1]$, so this is a formula
 that becomes false once it has been announced. 


\end{document}
