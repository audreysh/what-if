% Part: applied-modal-logics
% Chapter: epistemic-logic
% Section: properties-accessibility

\documentclass[../../../include/open-logic-section]{subfiles}

\begin{document}

\olfileid{aml}{el}{acc}

\olsection{Accessibility Relations and Epistemic Principles}

Given what we already know about frame correspondence in normal modal
logics, we might want to see what the characteristic !!{formula}s look
like given epistemic interpretations. We have already said that
epistemic logics are typically interpreted in S5. So let's take a look
at how various epistemic principles are represented, and consider how
they correspond to various frame conditions.

Recall from normal modal logic, that different modal !!{formula}s characterized different properties of accessibility relations. This table picks out a few that correspond to particular epistemic principles.

\begin{table}[t]
    \begin{tabular}{| p{.48\textwidth} || p{.48\textwidth} |}
      \hline
      {\emph{If $R$ is \dots}} & {\emph{then \dots is true in~$\mModel{M}$:}} \\
      \hline \hline
        
      & $\Knows (p \lif q) \lif (\Knows p \lif \Knows q)$ 
      \hfill \newline{(Closure)} \\
      \hline
      \emph{reflexive}: $\forall w Rww$  
      & $\Knows p \lif p$ \hfill \newline{(Veridicality)} \\
      \hline
      \emph{transitive}: \newline
      $\forall u \forall v \forall w ((Ruv \land Rvw) \lif Ruw)$ & 
      $\Knows p \lif \Knows \Knows p$ \hfill 
           \newline{(Positive Introspection)} \\
      \hline 
      \emph{euclidean}: \newline
      $\forall w \forall u \forall v ((Rwu \land Rwv) \lif Ruv)$ &  
      $\lnot \Knows p \lif \Knows \neg \Knows p$ \hfill 
        \newline{(Negative Introspection)} \\
      \hline
    \end{tabular}
    \caption{Four epistemic principles.}
    \ollabel{tab:four}
  \end{table} 

Veridicality, corresponding to the $T$ axiom, is often treated as the most uncontroversial of these principles, as it represents that claim that if a !!{formula} is known, then it must be true. Closure, as well as Positive and Negative Introspection are much more contested. 

Closure, corresponding to the $K$ axiom, represents the idea that an agent's knowledge is closed under implication. This might seem plausible to us in some cases. For instance, I might know that if I am in Victoria, then I am on Vancouver Island. Barring odd skeptical scenarios, I do know that I am in Victoria, and this should also suggest that I know I am on Vancouver Island. So in this case, the logical closure of my knowledge might seem relatively intuitive. On the other hand, we do not always think through the consequences of our knowledge, and so this might lead to less intuitive results in other cases.

Positive Introspection, sometimes known as the KK-principle, is sometimes articulated as the statement that if I know something, then I know that I know. It is the epistemic counterpart of the 4 axiom. Correspondingly, negative introspection is articulated as the statement that if I \emph{don't} know something, then I know that I don't know it, which is the counterpart of the 5 axiom. Both of these seem to admit of relatively ordinary counterexamples, in which I am unsure whether or not I know something that I do in fact know. 


\end{document}
