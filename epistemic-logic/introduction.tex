% Part: applied-modal-logics
% Chapter: epistemic-logic
% Section: introduction

\documentclass[../../../include/open-logic-section]{subfiles}

\begin{document}

\olfileid{aml}{el}{int}

\olsection{Introduction}

Just as modal logic deals with \emph{modal propositions} and the entailment relations among them, epistemic logic deals with \emph{epistemic propositions} and the entailment relations among them. Rather than interpreting the modal operators as representing possibility and necessity, the unary connectives are interpreted in epistemic or doxastic ways, to model knowledge and belief. For example, we might want to express claims like the following:

\begin{enumerate}
\item Richard knows that Calgary is in Alberta.
\item Audrey thinks it is possible that a dog is on the couch.
\item Richard knows that Audrey knows that her class is on Tuesdays.
\item Everyone knows that a year has 12 months.
\end{enumerate}
Contemporary epistemic logic is often traced to Jaako Hintikka's \emph{Knowledge and Belief}, from 1962, and it was written at a time when possible worlds semantics were becoming increasingly more used in logic. In fact, epistemic logics use most of the same semantic tools as other modal logics, but will interpret them differently. The main change is in what we take the \emph{accessibility relation} to represent. In epistemic logics, they represent some form of \emph{epistemic possibility}. We'll see that the epistemic notion that we're modelling will affect the constraints that we want to place on the accessibility relation. And we'll also see what happens to correspondence theory when it is given an epistemic interpretation. You'll notice that the examples above mention two agents: Richard and Audrey, and the relationship between the things that each one knows. The epistemic logics we'll consider will be multi-agent logics, in which such things can be expressed. In contrast, a single-agent epistemic logic would only talk about what one individual knows or believes.

\end{document}
