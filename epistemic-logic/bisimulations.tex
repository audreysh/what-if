% Part: applied-modal-logics
% Chapter: epistemic-logic
% Section: bisimulations

\documentclass[../../../include/open-logic-section]{subfiles}

\begin{document}

\olfileid{aml}{el}{bsd}

\olsection{Bisimulations}

One remaining question that we might have about the expressive power
of our epistemic language has to do with the relationship between
models and the formulas that hold in them. We have seen from our frame
correspondence results that when certain formulas are valid in a
frame, they will also ensure that those frames satisfy certain
properties. But does our modal language, for example, allow us to
distinguish between a world at which there is a reflexive arrow, and
an infinite chain of worlds, each of which leads to the next? That is,
is there any formula $A$ that might hold at only one of these two
worlds?

Bisimulation is a relationship that we can define between relational
models to say that they have effectively the same structure. And as we
will see, it will capture a sense of equivalence between models that
can be captured in our epistemic language.

\begin{defn}[Bisimulation]
Let $M_1 = \langle W_1, R_1, V_1 \rangle$ and $M_2 = \langle W_2, R_2, V_2 \rangle$ be two relational models. And let $\mathcal{R} \subseteq W_1 \times W_2$ be a binary relation. We say that $\mathcal{R}$ is a \emph{bisimulation} when for every $\langle w_1, w_2 \rangle \in \mathcal{R}$, we have:

\begin{enumerate}
  \item {$w_1 \in V_1(p)$ iff $w_2 \in V_2(p)$ for all
  !!{propositional variable}s~$p$.}{}
  \item {For all agents $a \in A$ and worlds $v_1 \in W_1$, if
  $R_{1_a} w_1 v_1$ then there is some $v_2 \in W_2$ such that
  $R_{2_a} w_2 v_2$, and $\langle v_1, v_2 \rangle \in
  \mathcal{R}$.}{}
  \item {For all agents $a \in A$ and worlds $v_2 \in W_2$, if
   $R_{2_a} w_2 v_2$ then there is some $v_1 \in W_1$ such that
   $R_{1_a} w_1 v_1$, and $\langle v_1, v_2 \rangle \in
   \mathcal{R}$.}{}
\end{enumerate}

When there is a bisimulation between $M_1$ and $M_2$ that links worlds
$w_1$ and $w_2$, we can also write $(M_1, w_1) \leftrightarroweq (M_2,
w_2)$, and call $(M_1, w_2)$ and $(M_2, w_2)$ \emph{bisimilar}.
\end{defn}

The different clauses in the bisimulation relation ensure different
things. Clause 1 ensures that bisimilar worlds will satisfy the same
modal-free formulas, since it ensures agreement on all
!!{propositional variable}s. The other two clauses, sometimes referred
to as ``forth'' and ``back,'' respectively, ensure that the
accessibility relations will have the same structure.

\begin{thm}
If $(M_1, w_1) \leftrightarroweq (M_2, w_2)$, then for every formula
$A$, $M_1, w_1 \Vdash A$ iff $M_2, w_2 \Vdash A$.
\end{thm}

\begin{figure}
  \begin{center}
    \begin{tikzpicture}[modal]
      \node[world] (w1) {$w_1$}; 
      \node[world] (w2) [above left=of w1]{$w_2$}; 
      \node[world] (w3) [above right=of w1] {$w_3$};
      \draw[<->] (w1) to node {$a$} (w2);
      \draw[->,loop below] (w1) to node {$a$} (w1);
      \draw[<->] (w1) to [swap] node {$a$} (w3);
      \draw[->,loop above] (w2) to node {$a$} (w2) ;
      \draw[->,loop above] (w3) to node {$a$} (w3) ;
      
      \node[world](v1)[right of=w1, xshift=1.5in]{$v_1$};
      \node[world](v2)[above of=v1]{$v_2$};
      \draw[<->] (v1) to node {$a$} (v2);
      \draw[->,loop below] (v1) to node {$a$} (v1);
      \draw[->,loop above] (v2) to node {$a$} (v2) ;

      \draw[-,dotted] (w1) to (v1) ;
      \draw[-,dotted,bend left=45] (w2) to (v2) ;
      \draw[-,dotted,bend left=30] (w3) to (v2) ;
    \end{tikzpicture}
  \end{center}
  \caption{Two bisimilar models.}
  \ollabel{fig:bisimilar}
\end{figure}

Even though the two models pictured in \olref{fig:bisimilar} aren't
quite the same as each other, there is a bisimulation linking worlds
$w_1$ and $v_1$. This bisimulation will also link both $w_2$ and $w_3$
to $v_2$, with the idea being that there is nothing expressible in our
modal language that can really distinguish between them. The situation
would be different if $w_2$ and~$w_3$ satisfied different
!!{propositional variable}s, however. 

\end{document}
