% Part: applied-modal-logics
% Chapter: epistemic-logic
% Section: truth-at-w

\documentclass[../../../include/open-logic-section]{subfiles}

\begin{document}

\olfileid{aml}{el}{trw}

\olsection{Truth at a World}

Just as with normal modal logic, every epistemic model determines which !!{formula}s count as true at
which worlds in it. We use the same notation ``model $\mModel{M}$ makes
!!{formula}~$!A$ true at world~$w$'' for the basic notion of relational
semantics. The relation is defined inductively and is identical to the normal modal case for all non-modal operators.

\begin{defn}\ollabel{defn:mmodels}
  \emph{Truth of !!a{formula}~$!A$ at~$w$} in a~$\mModel M$, in symbols:
  $\mSat{M}{!A}[w]$, is defined inductively as follows:
  \begin{enumerate}
  \tagitem{prvFalse}{\indcase{!A}{\lfalse}{Never $\mSat{M}{\lfalse}[w]$}.}{}
  \tagitem{prvTrue}{\indcase{!A}{\ltrue}{Always $\mSat{M}{\ltrue}[w]$}.}{}
  \item $\mSat{M}{p}[w]$ iff $w \in V(p)$
  \tagitem{prvNot}{\indcase{!A}{\lnot !B}{$\mSat{M}{\indfrm}[w]$ iff
    $\mSat/{M}{!B}[w]$}.}{}
  \tagitem{prvAnd}{\indcase{!A}{(!B \land !C)}{$\mSat{M}{\indfrm}[w]$ iff
    $\mSat{M}{!B}[w]$ and $\mSat{M}{!C}[w]$}.}{}
  \tagitem{prvOr}{\indcase{!A}{(!B \lor !C)}{$\mSat{M}{\indfrm}[w]$ iff
    $\mSat{M}{!B}[w]$ or $\mSat{M}{!C}[w]$} (or both).}{}
  \tagitem{prvIf}{\indcase{!A}{(!B \lif !C)}{$\mSat{M}{\indfrm}[w]$ iff
    $\mSat/{M}{!B}[w]$ or $\mSat{M}{!C}[w]$}.}{}
  \tagitem{prvIff}{\indcase{!A}{(!B \liff !C)}{$\mSat{M}{\indfrm}[w]$ iff
    either both $\mSat{M}{!B}[w]$ and $\mSat{M}{!C}[w]$ or
    neither $\mSat{M}{!B}[w]$ nor $\mSat{M}{!C}[w]$}.}{}
  \item\ollabel{defn:sub:mmodels-box}
    \indcase{!A}{\Knows_a !B}{$\mSat{M}{\indfrm}[w]$ iff
    $\mSat{M}{!B}[w']$ for all $w' \in W$ with $R_a ww'$}
  \end{enumerate} 
\end{defn}

Here's where we need to think about restrictions on our accessibility
relations, though. After all, by clause~\olref{defn:sub:mmodels-box},
!!a{formula}~$\Knows_a !B$ is true at~$w$ whenever there are no~$w'$
with $wR_aw'$. This is the same clause as in normal modal logic; when
a world has no successors, all $\Box$-formulas are vacuously true
there. This seems extremely counterintuitive if we think about
$\Knows$ as representing \emph{knowledge}. After all, we tend to think
that there are \emph{no} circumstances under which an agent might know
both $!A$ and $\lnot !A$ at the same time.

One solution is to ensure that our accessibility relation in epistemic
logic will always be \emph{reflexive}. This roughly corresponds to the
idea that the actual world is consistent with an agent's information.
In fact, epistemic logics typically use S5, but others might use
weaker systems depending on what exactly they want the $\Knows_a$
relation to represent.

\begin{figure}
  \begin{center}
    \begin{tikzpicture}[modal]
      \node[world] (w1) [label={right:\mTrue{p}\\ \mFalse{q}}]{$w_1$}; 
      \node[world] (w2) [label={right:\mTrue{p}\\ \mTrue{q}},
        above right=of w1]{$w_2$}; 
      \node[world] (w3) [label={right:\mFalse{p}\\ \mFalse{q}},
        below right=of w1] {$w_3$};
      \draw[<->] (w1) to node {$a$} (w2);
      \draw[->,loop left] (w1) to node {$a,b$} (w1);
      \draw[<->] (w1) to [swap] node {$b$} (w3);
      \draw[->,loop above] (w2) to node {$a,b$} (w2) ;
      \draw[->,loop below] (w3) to node {$a,b$} (w3) ;
    \end{tikzpicture}
  \end{center}
  \caption{A simple epistemic model.}
  \ollabel{fig:simple}
\end{figure}

\begin{prob}
  Consider which of the following hold in
  \olref[aml][el][trw]{fig:simple}:
  \begin{enumerate}
    \item $\mSat{M}{\lnot q}[w_1]$;
    \item $\mSat{M}{\Knows_a \lnot q}[w_1]$;
    \item $\mSat{M}{\Knows_b \lnot q}[w_1]$;
    \item $\mSat{M}{\Knows_b q \lor \Knows_b \lnot q}[w_2]$;
    \item $\mSat{M}{\Knows_a( \Knows_b q \lor \Knows_b \lnot q)}[w_2]$;
    \item $\mSat{M}{\EKnows_{\{a,b\}} \lnot q}[w_3]$;
    \end{enumerate}
\end{prob}

Now that we have given our basic definition of truth at a world, the
other semantic concepts from normal modal logic, such as modal
validity and entailment, simply carry over, applied to this new way of
thinking about the interpretation for the modal operators.

We are now also in a position to give truth conditions for the common
knowledge operator~$\CKnows_G$. Recall from \olref[sfr][rel][ops]{sec}
that the \emph{transitive closure}~$R^+$ of a relation~$R$ is defined
as
\begin{align*}
  R^+ &= \bigcup_{ n \in \mathbb{N}} R^n, 
\intertext{where}
  R^0 & = R \text{ and}\\ 
  R^{n+1} & = \Setabs{\tuple{x, z}}{\exists y (R^n xy \land Ryz)}.
\end{align*}
Then, where $G$ is a group of agents, we define $R_G = ( \bigcup_{b
\in G} R_b )^+$ to be the transitive closure of the union of all
agents' accessibility relations.

\begin{defn}
If $G' \subseteq G$, we let $\mSat{M}{\CKnows_{G'} !A}[ w]$ iff for
every $w'$ such that $R_{G'} w w'$, $\mSat{M}{!A}[w']$.
\end{defn}

\end{document}
