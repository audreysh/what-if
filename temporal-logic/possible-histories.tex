% Part: applied-modal-logics
% Chapter: epistemic-logic
% Section: possible-histories

\documentclass[../../../include/open-logic-section]{subfiles}

\begin{document}

\olfileid{aml}{tl}{poss}

\olsection{Possible Histories}

The relational models of temporal logic that we have been using are
extremely flexible, since we do not have to place any restrictions on
the accessibility relation. This means that temporal models can branch
in the past and in the future, but we might want to consider a more
``modal'' conception of branching, in which we consider sequences of
events as possible histories. This does not necessarily require
changing our language, though we might also add our ``ordinary'' modal
operators $\Box$ and~$\Diamond$, and we could also consider adding
epistemic accessibility relations to represent changes in agents'
knowledge over time.

\begin{defn}
  A \emph{possible histories model} for the temporal language is a triple
  $\mModel{M} = \tuple{T, C, V}$, where
  \begin{enumerate}
    \item $T$ is a nonempty set, interpreted as states in time.
    \item $C$ is a set of computational paths, or \emph{possible
      histories} of a system. In other words, $C$~is a set of
      sequences~$\sigma$ of states $s_1$, $s_2$,~$s_3$, \dots, where
      every $s_i \in T$.
    \item $V$ is a function assigning to each !!{propositional
      variable}~$p$ a set~$V(p)$ of points in time.
  \end{enumerate}
  To make things simpler, we will also generally assume that when a
  history is in~$C$, then so are all of its suffixes. For example, if
  $s_1$, $s_2$,~$s_3$ is a sequence in~$C$, then so are $s_2$, $s_3$
  and~$s_3$. Also, when two states $s_i$ and~$s_j$ appear in a
  sequence~$\sigma$, we say that $s_i \prec_\sigma s_j$ when $i < j$.
  When $t \in V(p)$ we say $p$~is \emph{true at}~$t$.
\end{defn}

The one relevant change is that when we evaluate the truth of
!!a{formula} at a point in time~$t$ in a model~$\mModel M$, we do so
relative to a history~$\sigma$, in which $t$ appears as a state. We do
not need to change any of the semantics for !!{propositional
variable}s or for truth-functional connectives, though. All of those
are exactly as they were in \olref[sem]{defn:tmodels}, since none of
those will make reference to~$\sigma$. However, we now redefine our
future operator~$\Ftemp$ and add our~$\Diamond$ operator with respect
to these histories. 

\begin{defn}\ollabel{defn:phmodels}
  \emph{Truth of !!a{formula}~$!A$ at~$t, \sigma$} in~$\mModel M$, in symbols:
  $\mSat{M}{!A}[t, \sigma]$:
  \begin{enumerate}
  \item\ollabel{defn:sub:phmodels-f} \indcase{!A}{\Ftemp
    !B}{$\mSat{M}{\indfrm}[t, \sigma]$ iff $\mSat{M}{!B}[t', \sigma]$
    for some $t' \in T$ such that $t \prec_\sigma t'$.}
  \item\ollabel{defn:sub:phmodels-diamond} \indcase{!A}{\Diamond
    !B}{$\mSat{M}{\indfrm}[t, \sigma]$ iff $\mSat{M}{!B}[t, \sigma']$
    for some $\sigma' \in C$ in which $t$ occurs.}
  \end{enumerate} 
\end{defn}

Other temporal and modal operators can be defined similarly. However,
we can now represent claims that combine tense and modality. For
example, we might symbolize ``$p$~will not occur, but it might have
occurred'' using the formula $\lnot \Ftemp p \land \Diamond \Ftemp p$.
This would hold at a point and a history at which $p$ does not become
true at a successor state, but there is an alternative history at
which $p$ will become true. 

\end{document}
