% Part: applied-modal-logics
% Chapter: temporal-logic
% Section: properties-accessibility

\documentclass[../../../include/open-logic-section]{subfiles}

\begin{document}

\olfileid{aml}{tl}{acc}

\olsection{Properties of Temporal Frames}

Given that our temporal models do not impose any conditions on the
relation~$\prec$, the only one of our familiar axioms that holds in
all models is~$K$, or its analogues $K_{\Gtemp}$ and~$K_{\Htemp}$:
\begin{align*}
\tag{$K_{\Gtemp}$}  \Gtemp (p \to q) & \to (\Gtemp p \to \Gtemp q)\\
\tag{$K_{\Htemp}$}  \Htemp (p \to q) & \to (\Htemp p \to \Htemp q)
\end{align*}

However, if we want our models to impose stricter conditions on how
time is represented, for instance by ensuring that $\prec$ is a linear
order, then we will end up with other validities in our models.

\begin{table}[t]
  \begin{tabular}{| p{.48\textwidth} || p{.48\textwidth} |}
    \hline
    {\emph{If $\prec$ is \dots}} & {\emph{then \dots is true in~$\mModel{M}$:}} \\
    \hline \hline
    \emph{transitive}: \newline
    $\forall u \forall v \forall w ((u \prec v \land v \prec w) \lif u \prec w)$ & 
    $\Ftemp \Ftemp p \lif \Ftemp p$  \\
    \hline 
    \emph{linear}: \newline
    $\forall w \forall v (w \prec v \lor w = v \lor v \prec w)$ &  
    $(\Ftemp \Ptemp  p \lor \Ptemp \Ftemp p) \lif (\Ptemp p \lor p \lor \Ftemp p)$\\
    \hline
    \emph{dense}: \newline
    $\forall w \forall v (w \prec v \to \exists u(w \prec u \land u \prec v))$ &  
    $\Ftemp p \lif \Ftemp \Ftemp p$ \\
    \hline
    \emph{unbounded (past)}: \newline
    $\forall w \exists v( v \prec w)$ &  
    $\Htemp p \to \Ptemp p$ \\
    \hline
    \emph{unbounded (future)}: \newline
    $\forall w \exists v( w \prec v)$ &  
    $\Gtemp p \to \Ftemp p$ \\
    \hline
  \end{tabular}
  \caption{Some temporal frame correspondence properties.}
  \ollabel{tab:correspondence}
\end{table} 

Several of the properties from \olref{tab:correspondence} might seem
like desirable features for a model that is intended to represent
time. However, it is worth noting that, even though we can impose
whichever conditions we like on the $\prec$ relation, not all
conditions correspond to !!{formula}s that can be expressed in the
language of temporal logic. For example, irreflexivity, or the idea
that $\forall w \lnot (w \prec w)$, does not have a corresponding
formula in temporal logic. 

\end{document}
