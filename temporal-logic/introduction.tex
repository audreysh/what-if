% Part: applied-modal-logics
% Chapter: temporal-logic
% Section: introduction

\documentclass[../../../include/open-logic-section]{subfiles}

\begin{document}

\olfileid{aml}{tl}{int}

\olsection{Introduction}

Temporal logics deal with claims about things that will or have been the case. Arthur Prior is credited as the originator of temporal logic, which he called \emph{tense logic}. Our treatment of temporal logic here will largely follow Prior's original modal treatment of introducing temporal operators into the basic framework of propositional logic, which treats claims as generally lacking in tense.

For example, in propositional logic, I might talk about a dog, Beezie, who sometimes sits and sometimes doesn't sit, as dogs are wont to do. It would be contradictory in classical logic to claim that Beezie is sitting and also that Beezie is not sitting. But obviously both can be true, just not at the same time; adding temporal operators to the language can allow us to express that claim relatively easily. The addition of temporal operators also allows us to account for the validity of inferences like the one from ``Beezie will get a treat or a ball" to ``Beezie will get a treat or Beezie will get a ball." 

However, a lot of philosophical issues arise with temporal logic that might lead us to adopt one framework of temporal logic over another. For example, a future contingent is a statement about the future that is neither necessary nor impossible. If I say ``Richard will go to the grocery store tomorrow," I am expressing a claim about something that has not yet happened, and whose truth value is contestable. In fact, it is contestable whether that claim can even be \emph{assigned} a truth value in the first place. If we are strict determinists, then perhaps we can be comfortable with the idea that this sentence is in fact true or false, even before the event in question is supposed to take place---it just may be that we do not know its truth value yet. In contrast, we might believe in a genuinely open future, in which the truth values of future contingents are undetermined. 

As it turns out, a lot of these commitments about the structure and nature of time are built in to our choices of models and frameworks of temporal logics. For example, we might ask ourselves whether we should construct models in which time is linear, branching or even circular. We might have to make decisions about whether our temporal models will have beginning and end points, and whether time is to be represented using discrete instants or as a continuum. 

\end{document}
