% !TeX root = ../wi-screen.tex

\chapter{Introduction}

Classical logic is very useful, widely used, has a long
history, and is relatively simple.  But it has limitations: for
instance, it does not (and cannot) deal well with certain locutions of
natural language such as tense and subjunctive mood, nor with certain
constructions such as ``Audrey knows that $p$.'' It makes certain
assumptions, for instance that every sentence is either true or false
and never both.  It pronounces some formulas tautologies and some
arguments as valid, even though these tautologies and arguments
formalize arguments in English which some do not consider true or
valid, at least not obviously.  Thus it seems there are examples where
classical logic is not expressive enough, or even where classical
logic gets things wrong.

This book discusses some alternative, \emph{non-classical}
logics. These non-classical logics are either more expressive than
classical logic or have different tautologies or valid arguments. For
instance, temporal logic extends classical logic by operators that
express tense; conditional logics have an additional, different
conditional (``if---then'') that does not suffer from the so-called
paradoxes of the material conditional.  All of these logics
\emph{extend} classical logic by new operators or connectives, and
fall into the broad category of intensional logics. Other logics such
as many-valued, intuitionistic, and paraconsistent logics have the
same basic connectives as classical logic, but different infrences
count as valid.  In many-valued and intutionistic logic, for instance,
the law of excluded middle $!A \lor \lnot !A$ fails to hold; in
paraconsistent logic the inference \emph{ex contradictione quodlibet},
$\lfalse \Entails !A$ for arbitrary~$!A$.

After we review the basics of classical propositional logic in
\cref{part:classical}, we begin our discussion of non-classical logics
in \ref{part:many-valued}. There we will relax one assumption
classical logic makes: that everything is either true or false. There
are good reasons to think that some sentences of English are
neither---they have some intermediate truth value. Examples of this
are sentences involving vagueness (``Mary is rich''), sentences the
truth of which is not yet determined (``There will be a sea battle
tomorrow''), and---an important case for philosopers---sentences that
are paradoxical such as ``This sentence is false.'' One of the
earliest non-classical logics allow truth values in addition to the
classical ``true'' and ``false''; they are called \emph{many-valued}.
We cover them in \cref{part:many-valued}.

Modal logics are extensions of classical logic by the operators $\Box$
(``box'') and $\Diamond$ (``diamond''), which attach to !!{formula}s.
Intuitively, $\Box$ may be read as ``necessarily'' and $\Diamond$ as
``possibly,'' so $\Box p$ is ``$p$ is necessarily true'' and $\Diamond
p$ is ``$p$ is possibly true.'' As necessity and possibility are
fundamental metaphysical notions, modal logic is obviously of great
philosophical interest. It allows the formalization of metaphysical
principles such as ``$\Box p \lif p$'' (if $p$ is necessary, it is
true) or ``$\Diamond p \lif \Box\Diamond p$'' (if $p$ is possible,
it is necessarily possible).

For the logic which corresponds to the interpretation of $\Box$ as
``necessarily,'' this semantics is relatively simple: instead of
assigning truth values to !!{propositional variable}s, an
interpretation~$\mModel{M}$ assigns a set of ``worlds'' to
them---intuitively, those worlds~$w$ at which $p$ is interpreted as
true. On the basis of such an interpretation, we can define a
satisfaction relation. The definition of this satisfaction relation
makes $\Box !A$ satisfied at a world~$w$ iff $!A$ is satisfied at
\emph{all} worlds: $\mSat{M}{\Box !A}[w]$ iff $\mSat{M}{!A}[v]$ for
all worlds~$v$. This corresponds to Leibniz's idea that what's
necessarily true is what's true in every possible world.

``Necessarily'' is not the only way to interpret the $\Box$ operator,
but it is the standard one---``necessarily'' and ``possibly'' are the
so-called \emph{alethic} modalities. Other interpretations read $\Box$
as ``it is known (by some person~$A$) that,'' as ``some person $A$
believes that,'' ``it ought to be the case that,'' or ``it will always
be true that.'' These are epistemic, doxastic, deontic, and temporal
modalities, respectively. Different interpretations of $\Box$ will
make different !!{formula}s logically true, and pronounce different
inferences as valid. For instance, everything necessary and everything
known is true, so $\Box !A \lif !A$ is a logical truth on the alethic
and epistemic interpretations. By contrast, not everything believed
nor everything that ought to be the case actually is the case, so
$\Box !A \lif !A$ is not a logical truth on the doxastic or deontic
interpretations. We discuss modal logics in general in
\cref{part:modal1,part:modal2} and epistemic logics in particular in
\cref{part:epistemic}.

In order to deal with different interpretations of the modal
operators, the semantics is extended by a relation between worlds, the
so-called accessibility relation.  Then $\mSat{M}{\Box !A}[w]$ if
$\mSat{M}{!A}[v]$ for all worlds~$v$ which are accessible from~$w$.
The resulting semantics is very versatile and powerful, and the basic
idea can be used to provide semantic interpretations for logics based
on other intensional operators. One such logic is a close relative of
modal logic called temporal logic. Instead of having just one modality
$\Box$ (plus its dual $\Diamond$), it has \emph{temporal operators}
such as ``always $P$,'' ``$p$ will be true'', etc. We study these in
\cref{part:temporal}.

Whereas the material conditional is best read as an English indicative
conditional (``If $p$ is true then $q$ is true''), subjunctive
conditionals are in the subjunctive mood: ``if $p$ were true then $q$
would be true.'' While a material conditional with a false antecedent
is true, a subjunctive conditional need not be, e.g., ``if humans had
tails, they would be able to fly.'' In \cref{part:counterfactuals}, we
discuss logics of counterfactual coditionals.

Intuitionistic logic is a constructive logic based on L. E. J.
Brouwer's branch of constructive mathematics. Intuitionistic logic is
philosophically interesting for this reason---it plays an important
role in constructive accounts of mathematics---but was also proposed
as a logic superior to classical logic by the influential English
philosopher Michael Dummett in the 20th century. As mentioned above,
intuitionistic logic is non-classical because it has fewer valid
inferences and theorems, e.g., $!A \lor \lnot !A$ and $\lnot\lnot !A
\lif !A$ fail in general. Intuitively, this is a consequence of the
intuitionist principle that something shouldn't count as true---you
should not assert it---unless you have a proof of it. And
obviously there are cases where we neither have a proof of~$!A$ nor a
proof of~$\lnot !A$. Intuitionistic logic can be given a relational
semantics very much like modal logic. We discuss it in
\cref{part:intuitionistic}.

One of the weirdest features of classical logic is the principle of
explosion: from a contradiction, anything follows. This principle
flows from the way we set up the semantics of classical logic, but it
is \emph{very} counterintuitive and goes against what we actually do
when we reason. After all, once you discover that some things you
believe are contradictory, you don't (usually) go on to conclude
arbitrary claims since they follow from your beliefs!{} This has led
logicians to develop systems of logic in which the inference \emph{ex
contradictione, quodlibet} is blocked. Some of these are simply
further weakenings of classical logic, designed just to get rid of
explosion. Some are more philosophically motivated. Part of what makes
the principle of explosion weird is that when it pronounces that $!A$
and $\lnot !A$ together entail $!B$, there is no connection at all
between the premises and the conclusion.  And shouldn't there be such
a connection in any valid argument? Shouldn't the premises be
\emph{relevant} to the conclusion?  This leads to so-valled
\emph{relevant} (or \emph{relevance}) logic. Another motivation is the
philosophical position called \emph{dialetheism:} the belief that
there can be true contradictions.  (If you believe that contradictions
can be true, then you would not want them to entail anything
whatsoever, e.g., something false.) All of these logic fall under the
umbrella term \emph{paraconsistent}. We discuss paraconsistent logics
in \cref{part:paraconsistent}.
